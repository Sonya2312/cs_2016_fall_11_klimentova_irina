    \documentclass[a4paper, 12pt]{book}
    \usepackage{comment}
    \usepackage{fullpage}
    \usepackage[utf8x]{inputenc}
    \usepackage[T2A]{fontenc}
    \usepackage[english,russian]{babel}
    \pagestyle{empty}

    \setcounter{chapter}{3}
    \setcounter{section}{2}

    \setcounter{subsection}{3}
    \setcounter{equation}{9}

    \usepackage[fleqn]{amsmath}

    \usepackage{color}
    \definecolor{light-gray}{rgb}{0.8,0.8,0.8}

    \begin{document}
    \noindent
    \large\textit{ 3.3 Estimating derivatives} \hfill \textbf{37} \\
    \\
    {\it What is a reasonable lumping rectangle?}\\
    \\
    The rectangle’s height is the peak height $n^n$ $e^{-n}$.
    For the rectangle’s width, use either the $1/e$ or
    the FWHM heuristic require approximating $t^n$ $e^{-t}$, expand its logarithm $f(t)$ in a Taylor series around its peak at $t=n$:
    \begin{equation}f=(n+\Delta t)=f(n)+\Delta t\frac{df}{dt}\Bigl|_{t=n}+\frac{(\Delta t)^2}{2}\frac{d^2f}{dt^2}\Bigl|_{t=n}+\ldots\end{equation}
    The second term of the Taylor expansion vanishes because $f(t)$ has zero slope at the peak. In the third term, the second derivative $d^2f/dt^2$ at $t=n$ is $-n/t^2$ or $-1/n$.Thus,
    \begin{equation}f=(n+\Delta t)\approx f(n)-\frac{(\Delta t)^2}{2n}\end{equation}
    To decrease $t^n$ $e^{-t}$ t by a factor of $F$ requires decreasing $f(t)$ by $\ln F$.This choice means $\sqrt{2n\ln F}$. Because the rectangle’s width is $2\Delta t$, the
    lumped-area estimate of ${n!}$ is
    \begin{equation}\\{n!}\sim n^n e^{-n} \sqrt{n}\times
    \begin{cases}
    \sqrt{8}~~~~~~~~\text{$(1/e~ criterion: F=e)$} \\
    \sqrt{8\ln 2}~~~\text{$(FWHM criterion: F=2)$}
    \end{cases}
    \end{equation}
    For comparison, Stirling’s formula is $n!\approx n^n e^{-n} \sqrt{2\pi n}$. Lumping has explained almost every factor. The $n^n e^{-n}$ factor is the height of the rectangle,
    and the $\sqrt{n}$ factor is from the width of the rectangle. Although the exact $\sqrt{2\pi}$ factor remains mysterious (Problem 3.9), it is approximated to within 13\% (the $1/e$ heuristic) or 6\% (the FWHM heuristic).\\

    \colorbox{light-gray}{
    \begin{minipage}{\textwidth}
    \large\textbf{Problem 3.8 Coincidence?}

    The FWHM approximation for the area under a Gaussian (Section 3.2.2) was also accurate to $6\%$ Coincidence?\\

    \large\textbf{Problem 3.9 Exact constant in Stirling’s formula}

    Where does the more accurate constant factor of $\sqrt{2\pi}$ come from?\end{minipage}}
    \\
    \section{Estimating derivatives}
    In the preceding examples, lumping helped estimate integrals. Because
    integration and differentiation are closely related, lumping also provides
    \newpage
    \noindent
    \large\textbf{38} \hfill \textit{3 Lumping} \\

    a method for estimating derivatives. The method begins with a dimensional observation about derivatives. A derivative is a ratio of differentials;for example,$df/dx$ is the ratio of $df$ or $dx$. Because $d$ is dimensionless
    (Section 1.3.2), the dimensions of $df/dx$ are the dimensions of f/x. This useful, surprising conclusion is worth testing with a familiar example:\\
    Differentiating height $y$ with respect to time $t$ produces velocity $dy/dt$, whose dimensions of $LT^{-1}$ are indeed the dimensions of $y/t$.
    \\
    \colorbox{light-gray}{
    \begin{minipage}{\textwidth}
    \textbf{Problem 3.10 Dimensions of a second derivative}

    What are the dimensions of $d^2f/dx^2$?\\
    \end{minipage}}
    \\
    \section{Secant approximation}
    As $df/dx$ and $f/x$ have identical dimensions,
    perhaps their magnitudes are similar:
    \begin{equation} \frac{df}{dx}\sim\frac{f}{x}\end{equation}
    Geometrically, the derivative $df/dx$ is the slope
    of the tangent line, whereas the approximation
    $f/x$ is the slope of the secant line. By replacing
    the curve with the secant line, we make a
    lumping approximation.
    Let’s test the approximation on an easy function such as $f(x)=x^2$. Good
    news—the secant and tangent slopes differ only by a factor of:
    \begin{equation}\frac{df}{dx}=2x~~and~~\frac{f(x)}{x}=x.\end{equation}

    \colorbox{light-gray}{
    \begin{minipage}{\textwidth}
    \textbf{Problem 3.11 Higher powers}
    Investigate
    the secant approximation for$f(x)=x^n$.
    Problem 3.12 Second derivatives
    Use the secant approximation to estimate $d^2f/dx^2$ with $f(x) = x^2$. How does the approximation compare to the exact second derivative?
    \end{minipage}}
    \\
    {\it How accurate is the secant approximation for $f(x)=x^2+100$ ?}\\
    \\
    The secant approximation is quick and useful but can make large errors.
    When $f(x)= x^2+100$ , for example, the secant and tangent at $x=1$
    \newpage
    \noindent
    \large\textit{3.3 Estimating derivatives} \hfill \textbf{39} \\

    have dramatically different slopes. The tangent slope $df/dx$ x is, whereas
    the secant slope $f(1)/1$ is $101$. The ratio of these two slopes, although dimensionless, is distressingly large.\\
    \colorbox{light-gray}{
    \begin{minipage}{\textwidth}
    \textbf{Problem 3.13 Investigating the discrepancy}
    With $f(x) = x^2 + 100$, sketch the ratio
    \begin{equation}\frac{secant slope}{tangent slope}\end{equation}
    as a function of x. The ratio is not constant! Why is the dimensionless factor not
    constant? (That question is tricky.)
    \end{minipage}} \\
    The large discrepancy in replacing the derivative $df/dx$, which is
    \begin{equation}\lim_{\Delta x\to0}\frac{f(x)-f(x-\Delta x)}{\Delta x},\end{equation}
    with the secant slope $f(x)/x$ is due to two approximations. The first
    approximation is to take $\Delta x=x$ rather than $\Delta x=0$ Then $ df/dx\approx
    (f(x) − f(0))/x$. This first approximation produces the slope of the line
    from $(0, f(0)) to (x, f(x))$. The second approximation replaces $f(0)$ with
    $0$, which produces $df/dx\approx f/x$; that ratio is the slope of the secant from $(0, 0)$ to $(x, f(x)).$ \\
    \section{Improved secant approximation}
    The second approximation is fixed by starting the secant at $(0, f(0))$ instead of $(0, 0).$ With that change, what are the secant and tangent slopes when $f(x) = x^2 + C$? Call the secant starting at $(0, 0)$ the origin secant; call the new secant the $x = 0$ secant.Then the $x = 0$ secant always has one-half the slope of the tangent, no matter the constant $C$. The $x = 0$ secant approximation is robust against—is unaffected by—vertical translation.
    \\
    {\it How robust is the x = 0 secant approximation against horizontal translation?}\\
    \\
    To investigate how the $x = 0$ secant handles horizontal translation, translate
    $f(x) = x^2$ rightward by $100$ to make $f(x)=(x−100)^2$. At the parabola’s
    \newpage
    \noindent
    \large\textbf{40} \hfill \textit{3 Lumping} \\

    vertex $x = 100$, the $x = 0$ secant, from $(0, 10^4)$ to $(100, 0)$, has slope $−100$;
    however, the tangent has zero slope. Thus the $x = 0$ secant, although an
    improvement on the origin secant, is affected by horizontal translation
    \\
    {\large\bf 3.3.3 Significant-change approximation}\\
    The derivative itself is unaffected by horizontal and vertical translation,
    so a derivative suitably approximated might be translation invariant. An
    approximate derivative is
    \begin{equation}\frac{df}{dx}\approx\frac{f(x+\Delta x)-f(x)}{\Delta x},\end{equation}
    where $\Delta x$ is not zero but is still small.
    \\
    {\it How small should $\Delta x$ be? Is $\Delta x = 0.01$ small enough?}\\
    \\
    The choice $\Delta x = 0.01$ has two defects. First, it cannot work when x has
    dimensions. If x is a length, what length is small enough? Choosing $\Delta x =
    1 mm$ is probably small enough for computing derivatives related to the
    solar system, but is probably too large for computing derivatives related
    to falling fog droplets. Second, no fixed choice can be scale invariant.
    Although $\Delta x = 0.01$ produces accurate derivatives when $f(x) = sin x$, it
    fails when $f(x) = sin 1000x$, the result of simply rescaling $x$ to $1000x$.
    \\
    {\large These problems suggest trying the following significant-change approximation:}\\
    \begin{equation}\frac{df}{dx}\approx\frac{\text{significant} \Delta x \text{(change in f)}\text{at} x}{\Delta x \text{that produces a significant} \Delta f}.\end{equation}
    Because the $\Delta x$ here is defined
    by the properties of the curve at the point
    of interest, without favoring particular coordinate values or values of $\Delta x$,
    the approximation is scale and translation invariant.
    To illustrate this approximation, let’s try $f(x) = cos x$ and estimate $df/dx$ at $x =
    3\pi/2$ with the three approximations: the origin secant, the $x = 0$ secant, and the significant-change approximation. The origin secant goes from $(0, 0)$ to $(3\pi/2, 0)$, so it has zero slope. It is a poor approximation to the exact slope of $1$. The $x = 0$
    \end{document}

