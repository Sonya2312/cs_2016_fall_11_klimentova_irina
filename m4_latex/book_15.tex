    \documentclass[a4paper, 11pt]{book}
    \usepackage{comment}
    \usepackage{fullpage}
    \usepackage[utf8x]{inputenc}
    \usepackage[T2A]{fontenc}
    \usepackage[english,russian]{babel}
    \pagestyle{empty}

    \setcounter{chapter}{3}
    \setcounter{section}{5}

    \setcounter{subsection}{3}
    \setcounter{equation}{47}

    \usepackage[fleqn]{amsmath}

    \usepackage{color}
    \definecolor{light-gray}{rgb}{0.8,0.8,0.8}

    \begin{document}
    \noindent
    \large\textit{3.5 Predicting the period of a pendulum} \hfill \textbf{53} \\

    Using the dimensionless period avoids writing the factors of $2\pi$, l, and g, and it yields the simple prediction
    \begin{equation}h(\theta_{0})\approx f(\theta_{0})^{{-1}/{2}}=\left( \frac{\sin\theta_{0}}{\theta_{0}} \right)^{{-1}/{2}}\end{equation}
    At moderate amplitudes the approximation closely
    follows the exact dimensionless period (dark curve). As a bonus, it also
    predicts $h({\pi}) ={\infty}$, so it agrees with the thought experiment of releasing
    the pendulum from upright (Section 3.5.3).

    \textit{How much larger than the period at zero amplitude is the period at $10\circ$ amplitude?}

    A 10$\circ$ amplitude is roughly 0.17 rad, a moderate angle, so the approximate
    prediction for h can itself accurately be approximated using a Taylor series.
    The Taylor series for $\sin\theta$ begins ${\theta}-{\theta^3/6}$, so
    \begin{equation}f(\theta_{0})=\frac{\sin\theta_{0}}{\theta_{0}}\approx 1-\frac{\theta_{0}^2}{6}\end{equation}
    Then $h(\theta_{0})$, which is roughly $f(\theta_{0})^{{-1}/{2}}$, becomes
    \begin{equation}h\theta_{0})\approx \left(1-\frac{\theta_{0}^2}{6} \right)^{{-1}/{2}}\end{equation}
    Another Taylor series yields $(1+x)^{{-1}/{2}}\approx 1 − x/2$ (for small x). Therefore,
    \begin{equation}h\theta_{0})\approx 1+\frac{\theta_{0}^2}{12}\end{equation}
    Restoring the dimensioned quantities gives the period itself.
    \begin{equation}T\approx 2\pi\sqrt{\frac{l}{g}}\left(1+\frac{\theta_{0}^2}{12}\right).\end{equation}
    Compared to the period at zero amplitude, a 10$\circ$ amplitude produces a
    fractional increase of roughly $\theta_{0}^2/12 \approx 0.0025$ or $0.25\%$
    amplitudes, the period is nearly independent of amplitude!

    \colorbox{light-gray}{
    \begin{minipage}{\textwidth}
    \textbf {Problem 3.5 Slope revisited}

    Use the preceding result for $h(\theta_{0})$ to check your conclusion in Problem 3.33
    about the slope of $h(\theta_{0})$ at $\theta_{0}= 0$.
    \end{minipage}
    }
    \newpage
    \noindent
    \large\textbf{54} \hfill \textit{3 Lumping} \\

    \textit{Does our lumping approximation underestimate or overestimate the period?}

    The lumping approximation simplified the pendulum differential equation
    by replacing $f(\theta)$ with $f(\theta_{0})$. Equivalently, it assumed that the mass
    always remained at the endpoints of the motion where $|\theta_{0}| = \theta_{0}$. Instead,
    the pendulum spends much of its time at intermediate positions where
    $|\theta| < \theta_{0}$ and $f(\theta) > f(\theta_{0})$. Therefore, the average f is greater than $(\theta_{0})$.
    Because h is inversely related to $f (h = f^\frac{-1}{2})$, the $\colon f(\theta) \to f(\theta_{0})$ lumping
    approximation overestimates h and the period.

    The $\colon f(\theta) \to f(\theta_{0})$ lumping approximation, which predicts $T= 2\pi\sqrt{\frac{l}{g}}$,
    underestimates the period. Therefore, the true coefficient of the $\theta_{0}^2$
    term
    in the period approximation
    \begin{equation}T\approx 2\pi\sqrt{\frac{l}{g}}\left(1+\frac{\theta_{0}^2}{12}\right)\end{equation}
    lies between 0 and 1/12. A natural guess is that the coefficient lies halfway
    between these extremes—namely, 1/24. However, the pendulum spends
    more time toward the extremes (where $f(\theta) = f(\theta_{0})$) than it spends near
    the equilibrium position (where $f(\theta)=f(\theta_{0})$). Therefore, the true coef-
    ficient is probably closer to 1/12—the prediction of the $\colon f(\theta) \to f(\theta_{0})$
    approximation—than it is to 0. An
    improved guess might be two-thirds
    of the way from 0 to 1/12, namely 1/18.

    In comparison, a full successive-approximation solution of the pendulum
    differential equation gives the following period [13, 33]:
    \begin{equation}T\approx 2\pi\sqrt{\frac{l}{g}}\left(1+\frac{\theta_{0}^2}{16}+\frac{11}{3072}\theta_{0}^4+...\right).\end{equation}
    Our educated guess of 1/18 is very close to the true coefficient of 1/16!

    \noindent
    \section[toc]{Summary and further problems}
    Lumping turns calculus on its head. Whereas calculus analyzes a changing
    process by dividing it into ever finer intervals, lumping simplifies a
    changing process by combining it into one unchanging process. It turns
    curves into straight lines, difficult integrals into multiplication, and mildly
    nonlinear differential equations into linear differential equations.
    \normalsize \textit{... the crooked shall be made straight, and the rough places plain. (Isaiah 40:4)}
    \newpage
    \noindent
    \large\textit{3.6 Summary and further problems} \hfill \textbf{55} \\

    \colorbox{light-gray}{
    \begin{minipage}{\textwidth}
    \large\textbf{Problem 3.36 FWHM for another decaying function}

    Use the FWHM heuristic to estimate
    \begin{equation}\int_{-\infty}^\infty\ \frac{dx}{1+x^4}\end{equation}
    Then compare the estimate with the exact value of $\pi/\sqrt{2}$. For an enjoyable
    additional problem, derive the exact value.

    \large\textbf{Problem 3.37 Hypothetical pendulum equation}

    Suppose the pendulum equation had been
    \begin{equation}\frac{d^2\theta}{d\theta^2} + \frac{l}{g}\tan \theta=0\end{equation}
    How would the period T depend on amplitude $\theta_{0}$? In particular, as $\theta_{0}$ increases,
    would T decrease, remain constant, or increase? What is the slope $dT/d\theta_{0}$ at
    zero amplitude? Compare your results with the results of Problem 3.33.

    For small but nonzero $\theta_{0}$, find an approximate expression for the dimensionless
    period $h(\theta_{0})$ and use it to check your previous conclusions.
    \large\textbf{Problem 3.38 Gaussian 1-sigma tail}
    The Gaussian probability density function with zero mean and unit variance is
    \begin{equation}p(x)=\frac{e^{-x^2/2}}{\sqrt{2\pi}}.\end{equation}
    The area of its tail is an important quantity in statistics, but it has no closed form.
    In this problem you estimate the area of the 1-sigma tail
    \begin{equation}\int_{1}^\infty\ \frac{e^{-x^2/2}}{\sqrt{2\pi}}dx=\frac{1-erf(1/\sqrt{2})}{2}\approx 1.59,\end{equation}
    where erf(z) is the error function.

    \large\textbf{Problem 3.37 Hypothetical pendulum equation}
    For the canonical probability Gaussian, estimate the area of its n-sigma tail (for
    large n). In other words, estimate
    \begin{equation}\int_{n}^\infty\ \frac{e^{-x^2/2}}{\sqrt{2\pi}} dx\end{equation}
    \end{minipage}
    }
    \end{document}

